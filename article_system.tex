% Ltex language=en
% \documentclass[12pt, 1p, review, number]{elsarticle}
\documentclass[3p,twocolumn,preprint]{elsarticle}
%\documentclass[10pt,a4paper,3p,twocolumn]{elsarticle}
% \documentclass[5p,preprint,twocolumn]{elsarticle}
%----------------------------------------------------------------------
%                     Loading of packages
%----------------------------------------------------------------------

% for cutting words
\usepackage[utf8]{inputenc}
% a font close to that chosen by Elsevier
\usepackage[bitstream-charter]{mathdesign}
\usepackage{mathtools, cuted}
% for accentuated characters
%\usepackage[T1]{fontenc}
\geometry{top=2cm,left=1.5cm,bottom=2cm,right=1.5cm}
\usepackage{multicol}
% For treating ps:
\usepackage{psfrag}
% For subfigures
\usepackage{graphicx}
\usepackage{subcaption}
\usepackage{placeins}
% For tables
\usepackage{array}
\usepackage{multirow}
\usepackage{hhline}
\usepackage{colortbl}
\usepackage{tabulary}
\usepackage{etoolbox}
% For citations
\usepackage[numbers]{natbib}
%\biboptions{longnamesfirst,angle,semicolon}
%\citetstyle{nature}
% For color
\usepackage{color}
\usepackage{xcolor}
% For spacing
\usepackage{setspace}
%\doublespacing
\singlespacing
%\onehalfspacing
% For maths
\usepackage{amsmath, amsfonts}
\usepackage[squaren,ashgrey]{SIunits}
% For Line numbering
\usepackage[pagewise,modulo]{lineno}
%\linenumbers
% For special characters
\usepackage{pifont}
%
\usepackage[pagebackref,colorlinks=True]{hyperref}
\usepackage{wasysym}
\usepackage{color}
\usepackage{cleveref}
\crefformat{figure}{(Fig.~#2#1#3)}
\usepackage{hyperref}
\usepackage[hyperpageref]{backref}
\usepackage{lipsum}
\usepackage{booktabs} % To thicken table lines
% \usepackage{widetext}
\usepackage{float}

%----------------------------------------------------------------------
%                     Commands
%----------------------------------------------------------------------
\renewcommand*{\overrightarrow}[1]{\vbox{\halign{##\cr \tiny\rightarrowfill\cr\noalign{\nointerlineskip\vskip1pt}$#1\mskip2mu$\cr}}}
%% Degree character
\DeclareTextSymbol{\degr}{T1}{6}
\newcommand{\degre}[0]{\degr C}
\DeclareUnicodeCharacter{2212}{-}
% MPam^0.5 character
\newcommand{\mpa}[0]{\mbox{MPa}\sqrt{\mbox{m}}}
%mettre de nouilles sous les tenseurs
\newcommand{\ve}[1]{{\underline{#1}}}
\def\ten#1{\oalign{$#1$\crcr\hidewidth$\scriptscriptstyle\sim$\hidewidth}} 
% derivee seconde
\newcommand{\fda}[2]{\displaystyle \frac{\partial\,{#1}}{\partial\, {#2}}}
% et al. for citations
\newcommand{\etal}[0]{{\em et al.}}
\renewcommand*\contentsname{\large{Table of contents}}
\def \hfillx {\hspace*{ -\linewidth} \hfill} % remplir espacement horizontal entre deux sous-figures de façon à remplir toute la largeur du texte.
%----------------------------------------------------------------------
%                     Colors
%----------------------------------------------------------------------
\definecolor{ashgrey}{rgb}{0.7, 0.75, 0.71}
%----------------------------------------------------------------------
%                     Pre-Document
%----------------------------------------------------------------------
\journal{Applied Energy}
%----------------------------------------------------------------------
%                     Document
%----------------------------------------------------------------------
\begin{document}

\tableofcontents

\begin{frontmatter}

%% Title, authors and addresses

%% use the tnoteref command within \title for footnotes;
%% use the tnotetext command for theassociated footnote;
%% use the fnref command within \author or \address for footnotes;
%% use the fntext command for theassociated footnote;
%% use the corref command within \author for corresponding author footnotes;
%% use the cortext command for theassociated footnote;
%% use the ead command for the email address,
%% and the form \ead[url] for the home page:
%% \title{Title\tnoteref{label1}}
%% \tnotetext[label1]{}
%% \author{Name\corref{cor1}\fnref{label2}}
%% \ead{email address}
%% \ead[url]{home page}
%% \fntext[label2]{}
%% \cortext[cor1]{}
%% \address{Address\fnref{label3}}
%% \fntext[label3]{}

\title{Ear canal dynamic motion piezoelectric energy harvester using a bistable oscillator cycled by coupled hydraulic switches made of collapsed flexible tubes.}

%% use optional labels to link authors explicitly to addresses:
%% \author[label1,label2]{}
%% \address[label1]{}
%% \address[label2]{}


\address[symme]{Laboratoire SYMME - Université Savoie Mont Blanc, 7 Chemin de Bellevue, 74940, Annecy}
\address[critias]{Laboratoire Critias - École de Technologie Supérieure, 1100 Rue Notre-Dame Ouest, Montréal, QC, H3C 1K3}


\author[symme]{Tigran AVETISSIAN}
\author[symme]{Fabien FORMOSA}
\author[symme]{Adrien BADEL}

\author[critias]{Michel DEMUYNCK}
\author[critias]{Aidin DELNAVAZ}
\author[critias]{Jérémy VOIX}


\begin{abstract}
%% Text of abstract
\end{abstract}

\begin{keyword}
%% Keywords
\end{keyword}

\end{frontmatter}

%\linenumbers

%% main text
%/!\/!\/!\/!\/!\/!\/!\/!\/!\/!\/!\/!\/!\/!\/!\/!\/!\/!\/!\/!\/!\/!\/!\/!\%
\section{INTRODUCTION}
\label{INTRODUCTION} % Ltex language=en
%/!\/!\/!\/!\/!\/!\/!\/!\/!\/!\/!\/!\/!\/!\/!\/!\/!\/!\/!\/!\/!\/!\/!\/!\%

The growing use of wireless devices and the miniaturization of electronic circuits have led to significant progress on the energy consumption of mobile devices around the human body. Energy harvesting methods have also been studied on purpose, in order to complement their power supply and enhance the autonomy of the batteries. In-ear devices such as hearing aids and cochlear implants are powered by disposable fuel cells or rechargeable batteries. The energy consumption of the best integrated devices in the literature converges around $17$J for a 10 hours daily use \cite{Scherer2019,Yip2015,Kulah2022}. Woodruff \emph{et al.} showed that patients using hearing aids sometimes struggle to change and select batteries for their devices \cite{Woodruff2021}. Another statistical study revealed that the majority of hearing aid and cochlear implants users prefer disposable batteries for the long autonomy, but would have liked rechargeable solutions if the battery cycle was sufficient to complete the day \cite{PracticesAudiology2016}. Also, the long term use makes the rechargeable solutions a more economical and more ecological choice compared to the disposable fuels. These arguments motivate the development of energy harvesting systems to enhance the autonomy of the batteries in order to prevent the premature discharge of in-ear devices.\\
The common sources are ambient such as solar or wind energy but their availability in the low size environment of the application context makes them not suitable for exploitation.
The activities of the human body generate a considerable amount of energy of different natures. Among them, the electrochemical energy from the inner ear \cite{Mercier2012}, the kinetic energy from walking or head movements \cite{Azimi2021,Smilek2016}, the body heat thermal energy \cite{Kim2014}, the strain energy from the skin deformation \cite{Jin2021} could be harvested for in-ear applications. The most relevant energy source for cochlear implants or hearing aids is the ear canal mechanical deformation as it is directly located in the area of the application. Delnavaz \emph{et al.} first showed in 2012 that the jaw movements changes de ear canal geometry during mastication. The figure \ref{fig:TMJ_on_earcanal} schematizes the mechanical interaction between the temporomandibular joint and the ear canal when the jaw is opened and when it is closed.
%%%%%%%%%%%%%%%%%%%%%%%%%%%%%%%%%%%%%%
\begin{figure}[!htbp]
	\centering
	\captionsetup{justification=centering}
	\includegraphics[trim={21cm 0cm 0cm 11.2cm},clip, width=0.9\linewidth]{figures/TMJ_on_earcanal.pdf}
	\caption{Mechanical interaction between the temporomandibular joint and the ear canal during jaw movements \cite{Delnavaz2012}}
	\label{fig:TMJ_on_earcanal}
\end{figure}
%%%%%%%%%%%%%%%%%%%%%%%%%%%%%%%%%%%%%%%
Carioli \emph{et al.} used a personalized earplug molding technic with a 3D scanner to reveal a global bending movement and a local compression area in the ear canal during mastication \cite{Carioli2016} (fig. \ref{fig:molding}). These two mechanical deformations represent an available energy source that can be exploited. This study first quantified the flexion energy amount to $5$mJ per mastication with a standard deviation of $4.9$mJ. The radial compression energy was also estimated to $1.3$mJ with $1.5$mJ of standard deviation. Besides this work, the literature exposes two technological solutions adapted to the extraction of the mechanical deformation energy in the ear canal. \\
%%%%%%%%%%%%%%%%%%%%%%%%%%%%%%%%%%%%%%
\begin{figure*}[!htbp]
	\centering
	\captionsetup{justification=centering}
	\includegraphics[trim={6cm 0cm 0cm 10cm},clip, width=0.6\linewidth]{figures/molding.pdf}
	\caption{(a) Earcanal in two extreme positions showing a global bending movement of the earcanal (b) Earcanal sectional view in two extreme positions showing a local compression area of the earcanal. \cite{Carioli2016}}
	\label{fig:molding}
\end{figure*}
%%%%%%%%%%%%%%%%%%%%%%%%%%%%%%%%%%%%%%%
Delnavaz \emph{et al.} first developed a flexible piezoelectric harvester made of a silicon earplug with a PVDF layer \cite{Delnavaz2013} (fig. \ref{fig:critias_piezo}). The earplug has been melted on the subject in order to maximize the contact surface with the ear canal. The harvester is capable of exploiting both the flexion and the compression energy sources. The experimental prototype presented on figure \ref{fig:critias_piezo} was able to extract $44$µJ from a normal mastication cycle. The soft materials facilitate the bio-integration of the harvester but the PVDF low electromechanical coupling coefficient results in an energy conversion efficiency below $1\%$. A hydroelectromagnetic energy harvester was also developed by the same researchers \cite{Delnavaz2012}. The figure \ref{fig:critias_emag} shows the harvester components and experimental test setup. The energy is extracted by a liquid filled earplug whose internal volume and pressure vary as a result of the radial compression of the ear canal internal walls during the jaw movements. These variations put into motion a moving magnet translating in a water column surrounded by a copper coil. The energy was harvested by electromagnetic induction and experimentally evaluated at 0.2µJ per mastication cycle. The hydraulic energy extraction limits the energy source to the radial compression as the ear canal global flexion does not generate any volume variation in the earplug. The low conversion efficiency ($<1\%$) is a consequence of the use of electromagnetic transduction for a low frequency energy source (1.57Hz) and a small scale application \cite{Kulah2008,Priya2017}. However, the hydraulic transmission interface allows the use of more complex and voluminous technological solutions since the energy becomes available outside the small ear canal volume. The table \ref{tab:harvesters_ear} summarizes the existing harvesters that exploit the ear canal mechanical deformation energy. \\
%%%%%%%%%%%%%%%%%%%%%%%%%%%%%%%%%%%%%%%
\begin{figure}[!htbp]
	\centering
	\captionsetup{justification=centering}
	\includegraphics[trim={18cm 0cm 0cm 8.7cm},clip, width=\linewidth]{figures/critias_piezo.pdf}
	\caption{Piezoelectric energy harvester. Experimental setup (dimensions are in mm). (a) Flattened PVDF layer. (b) PVDF layer. (c) Prototype of piezo-earpiece. (d) Piezo-earpiece mounted on the headset. (e) Measuring setup. \cite{Delnavaz2013}}  
	\label{fig:critias_piezo}
\end{figure}
%%%%%%%%%%%%%%%%%%%%%%%%%%%%%%%%%%%%%%%
%%%%%%%%%%%%%%%%%%%%%%%%%%%%%%%%%%%%%%%
\begin{figure}[!htbp]
	\centering
	\captionsetup{justification=centering}
	\includegraphics[trim={18cm 0cm 0cm 8.75cm},clip, width=\linewidth]{figures/critias_emag.pdf}
	\caption{Hydroelectromagnetic energy harvester. (a) Test setup. (b) Energy harvesting module. \cite{Delnavaz2012}} 
	\label{fig:critias_emag}
\end{figure}
%%%%%%%%%%%%%%%%%%%%%%%%%%%%%%%%%%%%%%%
%%%%%%%%%%%%%%%%%
\begin{table}[!htbp]
	\centering
	\captionsetup{justification=centering}
	\resizebox{\linewidth}{!}{%
	\begin{tabular}{ l  c  c  c}
		\toprule
		\multicolumn{1}{l}{\textbf{Harvester}}                        &
		\multicolumn{1}{c}{\textbf{Harvested energy per mastication}} &
		\multicolumn{1}{c}{\textbf{Reference}}                        \\
		\midrule
		Piezo earpiece        & $44$µJ 		 & \cite{Delnavaz2013} \\
	    Hydroelectromagnetic  & $0.2$µJ      & \cite{Delnavaz2012} \\
		Piezo sensor          & $0.1$µJ      & \cite{Carioli2018}  \\
		\bottomrule
	\end{tabular}}
	\caption{Existing harvesters exploiting the ear canal deformation energy}
	\label{tab:harvesters_ear}
\end{table}
%%%%%%%%%%%%%%%%%%%%%  
Based on the literature, we propose a new technological solution optimizing the energy conversion efficiency to maximize the harvested energy from the ear canal dynamic motion. The next section states on the energy source characterization and on the harvesting strategy we adopted. Section \ref{sec:HARVESTER PRESENTATION AND OPERATION PRINCIPLE} presents the harvester and describes its operation principle. Section \ref{sec:SYSTEM MODELING AND SIMULATIONS} shows the global system  modeling and the numerical simulation on its behavior. Then section \ref{sec:EXPERIMENTAL CHARACTERIZATIONS} then exposes the experimental validation performed on the harvester critical components and section \ref{sec:MODEL RECALIBRATION WITH EXPERIMENTAL DATA} presents the experimentally adjusted global harvester model. Section \ref{sec:ANALYZE AND DISCUSSION} discusses this work major contributions and the harvester potential. Finally, a conclusion is stated in the last section.

%/!\/!\/!\/!\/!\/!\/!\/!\/!\/!\/!\/!\/!\/!\/!\/!\/!\/!\/!\/!\/!\/!\/!\/!\%
\section{ENERGY SOURCE AND HARVESTING STRATEGY}
\label{sec:THE ENERGY SOURCE AND HARVESTING STRATEGY}
%/!\/!\/!\/!\/!\/!\/!\/!\/!\/!\/!\/!\/!\/!\/!\/!\/!\/!\/!\/!\/!\/!\/!\/!\%
    %///////////////////////////////////////////// 
	\subsection{The energy source}	
	\label{The energy source}
    %/////////////////////////////////////////////
The ear canal deflection energy is not a usual source because of the soft tissues and the low frequency of mastication. Thus, the traditional transducing methods are not suitable enough for the application.

On one hand, the ear canal global bending energy is 5 times higher than the local compression energy. To date, the literature proposes only one way to harvest the bending energy from the ear canal wall, which is to use of soft materials as the piezoelectric earplug previously introduced \cite{Delnavaz2013}. The very low mechanical impedance of the soft tissue in fact limits the use of more effective rigid piezoelectric materials.

On the other hand, the hydraulic extraction strategy proposed in \cite{Delnavaz2012} only exploits the local compression energy. However, it allows to transmit the energy outside the ear canal and benefit so from larger effective space. The latter is in fact essential as it becomes then possible to implement an impedance adaption stage and more effective transducing methods that could not fit directly inside the ear canal restricted volume. The present work is then based on the exploitation of the ear canal by a liquid filled earplug. As energy source, the latter can be assimilated to a micro pump characterized by its volume variation $V_{ear}$ and pressure variation $p_{ear}$. In order to optimize the energy extraction, the earplug must fit the ear canal wall. Turcot \emph{et al.} demonstrated that the initial pressurization must be at a minimum of 14kPa to achieve a good fit. Bouchard-Roy \emph{et al.} studied the dynamic pressure variation inside the earplug for different initial pressurization levels \cite{Bouchard-Roy2020}. A maximum of $\text{max}(\Delta p_{ear})=12$kPa of dynamic pressure variation , with a $28$kPa of initial pressurization, has been noted during the mastication for seven subjects. Besides, the temporal evolution of the volume variation $\Delta V_{ear}$ has been estimated in \cite{Delnavaz2012} and the result for one mastication cycle is presented on figure \ref{fig:deltaV_ear}.
%%%%%%%%%%%%%%%%%%%%%%%%%%%%%%%%%%%%%%
\begin{figure}[!htbp]
	\centering
	\captionsetup{justification=centering}
	\includegraphics[trim={20.5cm 0cm 0cm 10.8cm},clip, width=0.4\textwidth]{figures/deltaV_ear.pdf}
	\caption{Ear canal volume variation for one mastication cycle \cite{Delnavaz2012}}
	\label{fig:deltaV_ear}
\end{figure}
%%%%%%%%%%%%%%%%%%%%%%%%%%%%%%%%%%%%%%

The earplug operation characteristic behavior has not been studied yet. Thus, the evolution of the volume variation $\Delta V_{ear}$ is not well known as a function of the pressure variation $\Delta p_{ear}$. However, considering the pressure and volume variation levels from \cite{Delnavaz2012} and \cite{Bouchard-Roy2020}, the equation \ref{eq:max_hydraulic_energy} approximates an upper limit for the extractable hydraulic energy with a 28kPa pressurized earplug and a maximum volume variation of $\text{max}(\Delta V_{ear})=60$mm$^3$. The estimation assumes that the earplug is a perfect flow rate source, which is an optimistic hypothesis leading to $0.9$mJ available from one mastication. Based on 2200 mastication cycles per day, according to \cite{Goll2011}, there would be $2$mJ available per day, per ear. This amount of energy could theoretically enhance supply 23.5\% of the 17J of needed for the state of the art of cochlear implants \cite{Kulah2022}. Further research in both domains related to hearing aids and energy harvesting could lead to fully autonomous devices in the future.
\begin{align}
	\text{max}(E_{hyd}) = \text{max}(\Delta p_{ear}) \cdot \text{max}(\Delta V_{ear}) = 0.9\text{mJ}
	\label{eq:max_hydraulic_energy}
\end{align}

    %///////////////////////////////////////////// 
	\subsection{The harvesting strategy}	
	\label{The harvesting strategy}
    %/////////////////////////////////////////////
The application is not suitable for magnetic interactions for two major reasons. Magnetic fields admit high non-linear properties at small scales and their effectiveness decreases as the effective volume of transducing material and the frequency of the energy source decrease \cite{Priya2017,Kulah2008}. The piezoelectric ceramics yet appear like good candidates, but they can not be easily used because of their high stiffness. For this reason, and also because of the higher conversion efficiency of resonant structures, the harvesting system could benefit from a frequency-up conversion stage \cite{Ashraf2011,Peng2021}. These structures are commonly used for broadband energy harvesting. A majority of the papers with piezoelectric transducers use mechanical stops with piezoelectric ceramic cantilevers to reach high frequencies with a good mechanical coupling \cite{Edwards2013,Gu2011,Lee2007}. Stoppers however dissipate energy and demand maintenance with time, in order to replace rubbing parts. To overcome this drawback, alternative solutions use mechanically bistable oscillators (BO) to perform frequency-up without any rubbing parts \cite{Vocca2012}. Recent advances 
led to the development and optimization of  bistable structures integrating stacked piezoelectric ceramics inside flextensional elastic structures \cite{Huguet2017}. They show enhanced performances due to high electromechanical coupling and broadband efficiency.
Preliminary calculations based on the Duffing equation defining the BO mechanical behavior \cite{Tseng1971}, and the linearity of piezoelectric beams, offer a comparison tool to determine the best technological approach to maximize the harvested energy from the hydraulic source. Figure \ref{fig:OB_vs_LINEAR} shows in blue the hydraulic behavior of the ideal flow rate source previously introduced. To be harvested, the energy has to be stored in a mechanical potential reservoir at first, to be then converted in electricity by the transducer. The hydraulic-mechanic interface is chosen to be a hydraulic cylinder. On one hand, the best linear spring is capable of storing/extracting a maximum of $0.5\text{max}(E_{hyd})$ if actuated by the source through a hydraulic cylinder. On the other hand, a BO type non-linear spring is capable of extracting more than $0.65\text{max}(E_{hyd})$ from the same source.
%%%%%%%%%%%%%%%%%%%%%%%%%%%%%%%%%%%%%%
\begin{figure}[!htbp]
	\centering
	\captionsetup{justification=centering}
	\includegraphics[trim={22.5cm 0cm 0cm 8.8cm},clip, width=0.35\textwidth]{figures/OB_vs_LINEAR.pdf}
	\caption{Hydraulic source exploitation depending on the mechanical load}
	\label{fig:OB_vs_LINEAR}
\end{figure}
%%%%%%%%%%%%%%%%%%%%%%%%%%%%%%%%%%%%%%
Hence, the absence of rubbing parts and the optimal energy extraction shed light on a BO as a promising solution for our application. 
%/!\/!\/!\/!\/!\/!\/!\/!\/!\/!\/!\/!\/!\/!\/!\/!\/!\/!\/!\/!\/!\/!\/!\/!\%
\section{HARVESTER PRESENTATION AND OPERATION \mbox{PRINCIPLE}}
\label{sec:HARVESTER PRESENTATION AND OPERATION PRINCIPLE}
%/!\/!\/!\/!\/!\/!\/!\/!\/!\/!\/!\/!\/!\/!\/!\/!\/!\/!\/!\/!\/!\/!\/!\/!\%

%%%%%%%%%%%%%%%%%%%%%%%%%%%%%%%%%%%%%%%
\begin{figure*}[!htbp]
	\centering
	\captionsetup{justification=centering}
	\includegraphics[trim={3.2cm 0cm 0cm 4.3cm},clip, width=0.7\textwidth]{figures/system_presentation.pdf}
	\caption{Schematic presentation of the frequency-up piezoelectric harvester exploiting the ear canal geometry variation} 
	\label{fig:system_presentation}
\end{figure*}
%%%%%%%%%%%%%%%%%%%%%%%%%%%%%%%%%%%%%%%
The figure \ref{fig:system_presentation} presents a schematic view of the energy harvesting system. It uses a liquid filled earplug to transmit the energy outside the ear canal and benefit so from larger effective space. To maximize the efficiency, the system integrates a frequency-up conversion stage performed by a bistable oscillator (BO). The latter implements an amplified piezoelectric generator (APG) made of a flextensional elastic APX4 steel structure containing stacked lead zirconate titanate (PZT) ceramics (fig. \ref{fig:APG}). The electromechanical converter composed of the BO and the APG theoretically offers higher energy conversion efficiency compared to low frequency solutions and/or flexible electromechanical transducers \cite{Guo2019}.
%%%%%%%%%%%%%%%%%%%%%%%%%%%%%%%%%%%%	
\begin{figure}[!htb]
	\begin{center}
		\begin{subfigure}[t]{0.6\linewidth}
			\captionsetup{justification=centering}
			\includegraphics[trim={16cm 0cm 0cm 6cm},clip,width=\linewidth]{figures/APG_schema.pdf}
			\caption{Detailed schema}
			\label{fig:APG_schema}
		\end{subfigure}
		\hfillx
		\begin{subfigure}[t]{0.29\linewidth}
			\captionsetup{justification=centering}
			\includegraphics[trim={29.5cm 0cm 0cm 8cm},clip,width=0.65\linewidth]{figures/APG_photo.pdf}
			\caption{Picture}
			\label{fig:APG_photo}
		\end{subfigure}
		\caption{The APG presentation}
		\label{fig:APG}
	\end{center}
\end{figure}
%%%%%%%%%%%%%%%%%%%%%%%%%%%%%%%%%%%% 

The electromechanical converter is actuated, by two hydraulic cylinders (HC), with the hydraulic energy transmitted from the liquid filled earplug, through a pressure separator, a hydraulic amplifier and two hydraulic valves (HV).

To maximize the mechanical energy transfer between the ear canal and the earplug, the latter is lightly pressurized. A minimum of 14kPa is needed in order to fit the earplug to the ear canal wall \cite{Delnavaz2013}. The pressure separator ensures that this initial pressurization does not interfere on the system operation. Then, during a mastication cycle, the TMJ compresses the ear canal wall and expels so the fluid outside the earplug. The HVs ensure to led the fluid alternatively to each HC in order to actuate the BO mass back and forward. The hydraulic amplifier adapts the cylinders stroke to the available earplug flow rate and also adapts the mechanical impedance of the BO to the mechanical impedance of the ear canal wall.

The BO mass has two stable equilibrium positions, $x_m = x_0$ and $x_m = -x_0$. The harvester operates on two major phases. The first consists on the BO mass actuation, by a HC, from one stable equilibrium position toward the unstable position at $x_m = 0$. During the second phase the mass goes oscillate around the opposite stable equilibrium position until it stops, while a portion of the oscillatory energy is converted in electricity by the GPA. The mass then moves backward during the next mastication, completing so the harvester cycle.
%%%%%%%%%%%%%%%%%%%%%%%%%%%%%%%%%%%%%%%
\begin{figure}[!htbp]
	\centering
	\captionsetup{justification=centering}
	\includegraphics[trim={20cm 0cm 0cm 10.75cm},clip, width=\linewidth]{figures/HV_actuation_detail.pdf}
	\caption{Contact detail between a HV and the BO mass} 
	\label{fig:HV_actuation_detail}
\end{figure}
%%%%%%%%%%%%%%%%%%%%%%%%%%%%%%%%%%%%%%%

In order to maximize the harvested energy we must minimize the energy consumed by the HVs. A technological solution for this role is to use a flexible tube buckled by bending. The figure \ref{fig:HV_actuation_detail} schematizes the integration of such HV around the BO environment. The valve is composed of a flexible tube connected between a rigid part of the hydraulic circuit and a fixed one. The BO mass motion on the mobile part induces a bending angle $\theta$ around the rotational point $O$ located at the buckled section of the flexible tube. Consequently, a pressure loss is generated through the valve in buckled position. Thus, a HV is considered closed when the BO mass is at a stable equilibrium position ($x_m=\pm x_0$) and opened when the mass is at $x_m=0$. Two HVs symmetrically placed on each side of the $\vec{x}$ axis set up an opened hydraulic circuit on one side and a closed one on the other side, depending on the BO mass position. The HV technological choice is motivated by three major arguments. First, the absence of electromechanical transduction minimizes the energy losses for the valve operation. Then, the post-buckling softening of the bent tube minimizes the mechanical impact on the BO mass dynamic operation. Finally, it is well adapted at the millimetric application scale.

%/!\/!\/!\/!\/!\/!\/!\/!\/!\/!\/!\/!\/!\/!\/!\/!\/!\/!\/!\/!\/!\/!\/!\/!\%
\section{SYSTEM GLOBAL MODELING AND SIMULATIONS}
\label{sec:SYSTEM MODELING AND SIMULATIONS}
%/!\/!\/!\/!\/!\/!\/!\/!\/!\/!\/!\/!\/!\/!\/!\/!\/!\/!\/!\/!\/!\/!\/!\/!\%
The harvester is a multyphisic system composed of hydraulic, mechanical, and electrical components. This section presents a global multiphysic model describing the theoretical system operation. Two coupled subsystems are considered. The subsection \ref{subsec:The electromechanical converter} shows the modeling of the electromechanical frequency-up converter (BO + APG) under the mechanical influence of a HV and a HC. Then, the subsection \ref{subsec:The hydraulic circuit} discusses the design of the hydraulic circuit composed of the two HVs, the two HCs, the hydraulic amplifier, the pressure separator and the liquid filled earplug. Finally, the section \ref{subsec:Global model simulation} exposes the numerical simulation of the designed multiphysic coupled model.

    %///////////////////////////////////////////// 
	\subsection{Modeling of the electromechanical converter}	
	\label{subsec:The electromechanical converter}
    %/////////////////////////////////////////////
%%%%%%%%%%%%%%%%%%%%%%%%%%%%%%%%%%%%%%
\begin{figure*}[!htbp]
	\centering
	\captionsetup{justification=centering}
	\includegraphics[trim={0cm 0cm 0cm 0cm},clip, width=0.9\textwidth]{figures/schema_cinematique1.pdf}
	\caption{Cinematic schema of the of the electromechanical converter under the mechanical influence of a HV and a HC}
	\label{fig:schema_cinematique1}
\end{figure*}
%%%%%%%%%%%%%%%%%%%%%%%%%%%%%%%%%%%%%%

The figure \ref{fig:schema_cinematique1} shows the cinematic schema of the electromechanical converter under the mechanical influence of a HV and a HC. The different schematized rigid bodies all designated in the table \ref{tab:Designation of rigid bodies}. The BO is represented by 4 identical arms articulated by 8 hinges and a central mass. The APG is considered as a spring of stiffness $K$ with an electromechanical coupling $k^2$, indicative of the piezoelectric transduction, expressed by the equation \ref{eq:k2_def}.\\
The model is based on the following hypotheses:
\begin{itemize}
	\item The only mass considered is the BO mass (2).
	\item All parts are infinitely rigid beside the APG (4).
	\item The hinges are considered elastic. They are defined by their rotational stiffness $K_{\varphi}$ and their mechanical damping is included in the global viscous damping coefficient $\mu$.
	\item The contact between the BO mass (2) and the HC piston head (5) is considered permanent.
	\item We assume the small-angle approximation with \mbox{$x_0<<L$}. 
\end{itemize}
\begin{equation}
	k^2 = \dfrac{\alpha^2}{\alpha^2 +  K C_p}
	\label{eq:k2_def}
\end{equation}
%%%%%%%%%%%%%%%%%
\begin{table}[!htbp]
\centering
	\begin{tabular}{ c | c }
		\toprule
		\multicolumn{1}{c}{\textbf{Num}}  &
		\multicolumn{1}{c}{\textbf{Name}}  \\
		\midrule
		0	& Fixed frame      	  	\\  
		1	& BO arm      	  	\\  
		2	& BO mass     	\\
		3	& HC piston head	\\
		4	& APG     	\\
		5	& Simplified HV mechanical model     	\\
		\bottomrule
	\end{tabular}
	\caption{Designation of rigid bodies}
\label{tab:Designation of rigid bodies}
\end{table} 
%%%%%%%%%%%%%%%%%%%%%  

By isolating the BO mass and using the Newtons 2$^{nd}$ law, we can express its dynamic equilibrium with respect to the different external forces. The projection on the $\vec{x}$ axis of the resulting expression is given by the equation \ref{eq:OB-GPA} where the provenance of the different terms is highlighted. The BO mechanical equilibrium took aside contains the mass inertial term, a non-linear stiffness depending on $x_m$, the hinges stiffness and viscous losses. This expression has also been demonstrated by the Lagrange method in past works \cite{Liu2013}. The first term introduced by the HV is related to its rotationnal stiffness and the second term is related to the friction losses at the contact point $T$ (fig. \ref{fig:HV_actuation_detail}). A dry friction model is choose to describe the nature of the contact between the two components. Thus, $F_f$ will depend on a dry friction coefficient $\mu_f$, related to the nature of the material that are rubbing, and the sign of $\dot{x}_m$ (eq. \ref{eq:Dry_friction}). The force $F_{pis}$ induced by the piston head of the HC on the OB mass will depend on hydraulic parameters. The hydro-mechanical coupling is highlighted later by the equation \ref{eq:equilibre_dynamique_piston_ouvert}.
\begin{equation}
	F_f = \mu_f \text{sign}(\dot{x}_m)
	\label{eq:Dry_friction}
\end{equation}

The electrical power generated by the APG is dissipated in a load resistor $R_{ch}$ calculated with an adaptive impedance matching expressed by the equation \ref{eq:R_ch_adaptive_imp} \cite{Liu2013}. 
\begin{equation}
	R_{ch} = \dfrac{1}{C_p w_0}
	\label{eq:R_ch_adaptive_imp}
\end{equation}
Thus, the equation governing the electrical behavior of the APG is :
\begin{equation}
	\dfrac{U_p}{R_{ch}} = 
	-\alpha\dfrac{d}{dt}\biggl(2\sqrt{{l_0}^2-x_m^2}\biggr)
	- C_p\dot{U_p}
	= \frac{2\alpha x_m\dot{x}_m}{\sqrt{L^2+x_m^2}} - C_p\dot{U_p}
\end{equation}
Consequently, the generated power $P_e$ expression is :
\begin{equation}
	P_e = \frac{{U_p}^2}{R_{ch}} 
	\label{eq:P_e}
\end{equation} 
With the small-angle approximation, the system equation \ref{eq:OB+GPA+VH+piston} defines the electromechanical behavior of the harvester. The HV stiffness $K_{HV}$ will depend on its bending angle $\theta$ and the latter will geometrically depend on the BO mass position $x_m$, as showed in figure \ref{fig:HV_actuation_detail}.

%%%%%%%%%%%%%%%%%%%%%%%
\begin{figure*}
\begin{equation}
\overbrace{\ m \ddot{x}_m =-2K(\sqrt{x_m^2+L^2}-l_0)\tan(\varphi) 
							-\mu \dot{x}_m
							-\frac{4K_{\varphi}\varphi}{L}\ }^{OB}
			\underbrace{\ -2\alpha U_p \tan(\varphi)\ }_{APG}		 
			\overbrace{\ -\dfrac{K_{HV}\theta}{a} - F_f\ }^{HV}
			\underbrace{\ -F_{pis}\ }_{HC}
\label{eq:OB-GPA}
\end{equation}
\end{figure*}
% \FloatBarrier
%%%%%%%%%%%%%%%%%%%%%%%
%%%%%%%%%%%%%%%%%%%%%%%
\begin{figure*}
\begin{equation}
% \begin{small}
	\begin{dcases}
\ddot{x}_m = - \frac{2K{x_0}^2}{mL^2}\biggl(\frac{x_m^2}{{x_0}^2} -1\biggr)
			   x_m - \frac{\mu}{m}\dot{x}_m - \frac{2\alpha U_p\ x_m}{mL}
			-\frac{K_{VH}(x_m)\theta(x_m)}{ma}
			-\frac{F_f}{m}
			-\frac{F_{pis}}{m} \\
\dfrac{U_p}{R_{ch}} = \frac{2\alpha x_m \dot{x}_m}{L} - C_p\dot{U}_p
	\end{dcases}
	\label{eq:OB+GPA+VH+piston}	
% \end{small}
\end{equation}
\end{figure*}
% \FloatBarrier
%%%%%%%%%%%%%%%%%%%%%%%
    %///////////////////////////////////////////// 
	\subsection{Modeling of the hydraulic circuit}	
	\label{subsec:The hydraulic circuit}
    %/////////////////////////////////////////////
In order to model the coupled behavior of the two hydraulic branches, we will use the index "$o$" for the open branch and the index "$c$" for the closed one. The following hypotheses will be considered :
\begin{itemize}
	\item The fluid is incompressible and Newtonian.
	\item The hydraulic circuit is non-deformable (no volume change).
	\item There is no leakage.
	\item The hydraulic actuation is considered quasi-static (QS) in front
	the oscillation frequency of the BO.
\end{itemize}
The projection on the $\vec{x}$ axis of the Newtons 2$^nd$ law applied respectively to each HC moving piston gives us their mechanical equilibrium as follows :
\begin{align}
	\text{Opened cylinder ~}& \left\{~
	-F_{pis} - p_{oc}\ S_{c} - \lambda_{c}\ \dot{x}_{oc} = 0
	\right.
	\label{eq:equilibre_dynamique_piston_ouvert}\\
	\text{Closed cylinder ~}& \left\{~
	p_{cc}\ S_{c} - \lambda_{c}\ \dot{x}_{cc} = 0
	\right.
	\label{eq:equilibre_dynamique_piston_ferme}
\end{align}
Where $p_c$ is the HC chamber pressure and $x_c$ its moving piston position. $F_{pis}$ accounts here for the OB counter-reaction on the HC. Thus, with the QS assumption, it can be derived from equation \ref{eq:OB+GPA+VH+piston}	and expressed as follows : 
\begin{align}
	-F_{pis} & = F_{bo}\\
	-F_{pis} & = -\frac{K\ {x_0}^2}{L^2}\biggl(\frac{x_m^2}{{x_0}^2} -1\biggr)x_m - \biggl( \frac{K_{HV}\theta}{a} \biggr) - F_f
	\label{eq:F_OB_xxxxxx}
\end{align}

	%///////////////////////////////////////////// 
	\subsection{Global model simulation}	
	\label{subsec:Global model simulation}
	%/////////////////////////////////////////////




\newpage

%/!\/!\/!\/!\/!\/!\/!\/!\/!\/!\/!\/!\/!\/!\/!\/!\/!\/!\/!\/!\/!\/!\/!\/!\%
\label{NUMERICAL MODEL AND SIMULATIONS}
%/!\/!\/!\/!\/!\/!\/!\/!\/!\/!\/!\/!\/!\/!\/!\/!\/!\/!\/!\/!\/!\/!\/!\/!\%
%%%%%%%%%%%%%%%%%%%%%%%%%%%%%%%%%%%%%%
\begin{figure*}[!htbp]
	\centering
	\captionsetup{justification=centering}
	\includegraphics[trim={9.5cm 0cm 0cm 0cm},clip, width=0.7\textwidth]{figures/simu_pos_debit_Cf_pression_1CYCLE.pdf}
	\caption{$K_{HV}$ impact on the evolution of the BO mass potential energy.}
	\label{fig:simu_pos_debit_Cf_pression_1CYCLE}
\end{figure*}
%%%%%%%%%%%%%%%%%%%%%%%%%%%%%%%%%%%%%%%
%%%%%%%%%%%%%%%%%%%%%%%%%%%%%%%%%%%%%%
\begin{figure*}[!htbp]
	\centering
	\captionsetup{justification=centering}
	\includegraphics[trim={10cm 0cm 0cm 0cm},clip, width=0.7\textwidth]{figures/simu_pos_Up_puissances_energie_1CYCLE.pdf}
	\caption{$K_{HV}$ impact on the evolution of the BO mass potential energy.}
	\label{fig:simu_pos_Up_puissances_energie_1CYCLE}
\end{figure*}
%%%%%%%%%%%%%%%%%%%%%%%%%%%%%%%%%%%%%%%
%%%%%%%%%%%%%%%%%%%%%%%%%%%%%%%%%%%%%%
\begin{figure*}[!htbp]
	\centering
	\captionsetup{justification=centering}
	\includegraphics[trim={0cm 0cm 0cm 7cm},clip, width=0.8\textwidth]{figures/conversion_symme_rendements.pdf}
	\caption{$K_{HV}$ impact on the evolution of the BO mass potential energy.}
	\label{fig:conversion_symme_rendements.pdf}
\end{figure*}
%%%%%%%%%%%%%%%%%%%%%%%%%%%%%%%%%%%%%%%
%/!\/!\/!\/!\/!\/!\/!\/!\/!\/!\/!\/!\/!\/!\/!\/!\/!\/!\/!\/!\/!\/!\/!\/!\%
\section{EXPERIMENTAL CHARACTERIZATIONS}
\label{sec:EXPERIMENTAL CHARACTERIZATIONS}
%/!\/!\/!\/!\/!\/!\/!\/!\/!\/!\/!\/!\/!\/!\/!\/!\/!\/!\/!\/!\/!\/!\/!\/!\%
\lipsum[2]
    %///////////////////////////////////////////// 
	\subsection{The electromechanical converter}	
	\label{The electromechanical converter}
    %/////////////////////////////////////////////
%%%%%%%%%%%%%%%%%%%%%%%%%%%%%%%%%%%%%%
\begin{figure*}[!htbp]
\begin{center}
\captionsetup{justification=centering}
	\begin{subfigure} [h!]{0.49\textwidth}
		\includegraphics[trim={1.4cm 0cm 0cm 0cm},clip,width=\textwidth]{figures/monobloc+GPA_face.pdf}
		\caption{aaa} 
		\label{fig:/monobloc+GPA_face}
	\end{subfigure}
	%\hfill
	\begin{subfigure}[h!]{0.49\textwidth}
		\includegraphics[trim={6cm 3cm 7.5cm 2.6cm},clip, width=\textwidth]{figures/monobloc+GPA_iso.pdf}
		\caption{aaa}  
		\label{fig:/monobloc+GPA_iso}
	\end{subfigure}
	\caption{aaa}
\end{center}
\label{fig:/monobloc+GPA_face}
\end{figure*}
%%%%%%%%%%%%%%%%%%%%%%%%%%%%%%%%%%%%%%
\lipsum[1]
%%%%%%%%%%%%%%%%%%%%%%%%%%%%%%%%%%%%%%
\begin{figure}[!htbp]
	\centering
	\captionsetup{justification=centering}
	\includegraphics[trim={1cm 3cm 2cm 2.5cm},clip,width=0.45\textwidth]{figures/BDT_OB+GPA.pdf}
	\caption{Vue de face de GPA et l’OB monobloc sur le banc de caractérisation
	électromécanique.}
	\label{fig:BDT_OB+GPA}
\end{figure}
%%%%%%%%%%%%%%%%%%%%%%%%%%%%%%%%%%%%%%%
%%%%%%%%%%%%%%%%%
\begin{table}[!htbp]
\centering
\captionsetup{justification=centering}
	\begin{tabular}{ c | c | c }
	\toprule
	& \textbf{Simulation with}  	   & \textbf{Simulation with}        \\
	& \textbf{theoretical}  		   & \textbf{recalibrated }			 \\
	\multirow{-3}{*}{\textbf{Symbol}}
	& \textbf{parameters}			   & \textbf{parameters} 			 \\
	\midrule
	$Q$                       & 50.0                  & 30.0 		  	\\  
	$f_0$                     & 47.0 Hz               & 27.9 Hz  		\\
	$x_0$                     & 0.49 mm               & 0.50 mm    		\\
	$K$                       & 2.56e5 N/m            & 0.85e5 N/m 		\\
	${k^2_{sys}}$             & 8.72 \%               & 1.25 \% 		\\
	$\eta_{bo}$               & 79 \%                 & 12.9 \%   		\\
	\bottomrule
	\end{tabular}
	\caption{Theoretical and experimentally recalibrated values of the electromechanical converter.}
	\label{tab:parametres_lacher_free}
\end{table} 
%%%%%%%%%%%%%%%%%%%%%  

    %///////////////////////////////////////////// 
	\subsection{The hydraulic valves}	
	\label{The hydraulic valves}
    %/////////////////////////////////////////////
    		%************* 
			\subsubsection{Static characterizations}
			\label{Static characterizations of HV}
    		%************* 
	%%%%%%%%%%%%%%%%%%%%%%%%%%%%%%%%%%%%	
\begin{figure*}[!htb]
	\begin{center}
		\begin{subfigure}[t]{0.69\textwidth}
			\captionsetup{justification=centering}
			\includegraphics[trim={9.8cm 0cm 0cm 11cm},clip,width=\textwidth]{figures/essais_statique_VH.pdf}
			\caption{Vue globale}
			\label{fig:essais_statique_VH}
		\end{subfigure}
		\hfillx
		\begin{subfigure}[t]{0.29\textwidth}
			\captionsetup{justification=centering}
			\includegraphics[trim={14cm 0cm 0cm 7cm},clip,width=\textwidth]{figures/essais_statique_detail.pdf}
			\caption{Détail du contact support-échantillon}
			\label{fig:essais_statique_detail}
		\end{subfigure}
		\caption{Schéma des essais de caractérisations statiques de tubes en flexion}
		\label{fig:essais_statique_total}
	\end{center}
\end{figure*}
%%%%%%%%%%%%%%%%%%%%%%%%%%%%%%%%%%%% 
%%%%%%%%%%%%%%%%%%%%%%%%%%%%%%%%%%%%	
\begin{figure*}[!htb]
	\begin{center}
		\captionsetup{justification=centering}
		\includegraphics[trim={0cm 0cm 0cm 0cm},clip,width=\textwidth]{figures/resultats_essais_statique_VH_tous_sans_simu.pdf}
		\caption{Évolutions expérimentales de $K_t$ en fonction de $\theta$ pour quatre tubes de dimensions différentes (tab. \ref{tab:dim_tube_statique})}
		\label{fig:resultats_essais_statique_VH_tous}
	\end{center}
\end{figure*} 
%%%%%%%%%%%%%%%%%%%%%%%%%%%%%%%%%%%% 
    		%************* 
			\subsubsection{Hydraulic characterizations}
			\label{Hydraulic characterizations}
    		%************* 
%%%%%%%%%%%%%%%%%%%%%%%%%%%%%%%%%%%%	
\begin{figure*}[!htb]
\begin{center}
	\captionsetup{justification=centering} 
	\includegraphics[trim={8cm 0cm 0cm 5cm},clip,width=0.8\textwidth]{figures/BDT_hydraulique_VH.pdf}
	\caption{Banc d'essai pour la caractérisations hydrauliques de VH}
	\label{fig:BDT_hydraulique_VH}
\end{center}	
\end{figure*}    
%%%%%%%%%%%%%%%%%%%%%%%%%%%%%%%%%%%% 
%%%%%%%%%%%%%%%%%%%%%%%%%%%%%%%%%%%%	
\begin{figure*}[!htb]
\begin{center}
	\captionsetup{justification=centering} 
	\includegraphics[trim={2cm 0cm 0cm 10cm},clip,width=\textwidth]{figures/essais_hydraulique_VH.pdf}
	\caption{Essais de caractérisations hydrauliques de VH}
	\label{fig:essais_hydraulique_VH}
\end{center}	
\end{figure*}    
%%%%%%%%%%%%%%%%%%%%%%%%%%%%%%%%%%%%  
%%%%%%%%%%%%%%%%%%%%%%%%%%%%%%%%%%%%	
\begin{figure}[!htb]
\begin{center}
	\captionsetup{justification=centering} 
	\includegraphics[trim={2cm 0cm 0cm 4cm},clip,width=0.49\textwidth]{figures/fabrication_tube_experimental.pdf}
	\caption{Méthode de fabrication de la VH}
	\label{fig:fabrication_tube_experimental}
\end{center}	
\end{figure}    
%%%%%%%%%%%%%%%%%%%%%%%%%%%%%%%%%%%% 
%%%%%%%%%%%%%%%%%%%%%%%%%
\begin{figure}[!htbp]
\begin{center}
	\captionsetup{justification=centering}
	\includegraphics[trim={10cm 0cm 0cm 0cm},clip,width=0.49\textwidth]{figures/resultats_essais_hydraulique_VH_D1mm.pdf}
	\caption{$Cf_{VH}$ calculé à partir des essais pour 7 incréments d'angles}
	\label{fig:resultats_essais_hydraulique_VH_D1mm}
\end{center}
\end{figure}
%%%%%%%%%%%%%%%%
	%/////////////////////////////////////////////
	\subsection{The valve mechanical influence on the electromechanical converter}	
	\label{The mechanical influence of the valve on the electromechanical converter}
	%/////////////////////////////////////////////
%%%%%%%%%%%%%%%%%%%%%%%%%%%%%%%%%
\begin{figure*}[!htbp]
\begin{center}
	\begin{subfigure}[b]{0.49\textwidth}
	\captionsetup{justification=centering} 
	\includegraphics[trim={19cm 0cm 0cm 8cm},clip,width=\textwidth]{figures/presentation_BDT_avant_actionnement.pdf}
	\caption{Avant lâcher : $x_m=-x_0$}
	\label{fig:presentation_BDT_lacher_avant_actionnement}
	\end{subfigure}
\hfillx
	\begin{subfigure}[b]{0.49\textwidth}
	\captionsetup{justification=centering} 
	\includegraphics[trim={19cm 0cm 0cm 8cm},clip,width=\textwidth]{figures/presentation_BDT_apres_actionnement.pdf}
	\caption{Après lâcher : $x_m=x_{0,vh}$}
	\label{fig:presentation_BDT_lacher_apres_actionnement}
	\end{subfigure}
	\caption{Présentation du banc de test de lâcher avec VH}
	\label{fig:presentation_BDT_lacher_tube}
\end{center}	
\end{figure*} 
%%%%%%%%%%%%%%%%%%%%%%%%%%%%%%%%%
%%%%%%%%%%%%%%%%%%%%%%%%%%%%%%%%%%%%	
\begin{figure}[!htbp]
\begin{center}
	\captionsetup{justification=centering}
	\includegraphics[trim={4cm 0cm 0cm 9.5cm},clip,width=0.7\textwidth]{figures/contact_M_VH_lachers.pdf}
	\caption{Image de la VH$_{T100p}$ lorsque $\theta_0$($x_m=-x_{0})$ et lorsque $\theta_f$($x_m=x_{0,vh}$)}
	\label{fig:contact_M_VH_lachers}
\end{center}
\end{figure}
%%%%%%%%%%%%%%%%%%%%%%%%%%%%%%%%%%%%

%%%%%%%%%%%%%%%%%%%%%%%%%%%%%%%%%%%
\begin{table}[!htbp]
	\centering
	% \resizebox{\textwidth}{!}{%
		\begin{tabular}[t]{c|c|c}
\toprule
\multicolumn{1}{c}{\textbf{Paramètre}}	&
\multicolumn{1}{c}{\textbf{Valeur OB}} 	& 
\multicolumn{1}{c}{\textbf{Valeur OBVH}}  \\
\midrule
$D_g$ [mm] 						& \cellcolor{ashgrey} 		& \textcolor{red}{4} 		\\ \hline
$\Delta\theta$ [deg] 			& \cellcolor{ashgrey} 		& \textcolor{red}{{[$\approx$ 19 ; $\approx$ 36]}} \\ \hline
$K_{T100p}(\theta_f)$ [Nmm/rad] & \cellcolor{ashgrey}  		&  0.27 					\\ \hline
$a$ [mm]         			    & \cellcolor{ashgrey}  		&  2.24				 	 	\\ \hline
$\mu_{fs}$ [~~] 				& \cellcolor{ashgrey}  		&  0.42  					\\ \hline
$K$ [N/m] 						&	84480			  	 	&  84480  					\\ \hline
$x_0$ [mm] 						& \textcolor{red}{0.59}		& \textcolor{red}{0.50}  	\\ \hline
$v_0$ [mm/s] 					& \textcolor{red}{10}		& \textcolor{red}{65}  		\\ \hline
$Q$	[~~] 						& 		24.0		 		& 5.0     					\\ \hline
$k^2_{sys}$ [\%] 				& 		1.25		 		& \textcolor{red}{1.25}   	\\ \hline
$\eta_{ob}$ [\%] 				& 		11.9		 		& 2.6   					\\ \hline	
$R_{ch}$ [k$\text{\ohm}$] 		&	\textcolor{red}{15.5}	& \textcolor{red}{15.5}    	\\ \hline		
$m$	[g]						    &	\textcolor{red}{5.88}	& 9.00   					\\ \hline	
$f_0$ [Hz]						&		32.9				& 27.9   					\\
\bottomrule	
	\end{tabular}
        \caption{Valeur des paramètres de l'OB et de l'ensemble OBVH implémentant le tube T100p suite au essais de lâchers expérimentaux}
        \label{tab:parametres lacher tube}
\end{table}        
%%%%%%%%%%%%%%%%%%%%%%%%%%%%%%%%%%%%

%/!\/!\/!\/!\/!\/!\/!\/!\/!\/!\/!\/!\/!\/!\/!\/!\/!\/!\/!\/!\/!\/!\/!\/!\%
\section{MODEL RECALIBRATION WITH EXPERIMENTAL DATA}
\label{sec:MODEL RECALIBRATION WITH EXPERIMENTAL DATA}
\section{ANALYZE AND DISCUSSION}
\label{sec:ANALYZE AND DISCUSSION}
%/!\/!\/!\/!\/!\/!\/!\/!\/!\/!\/!\/!\/!\/!\/!\/!\/!\/!\/!\/!\/!\/!\/!\/!\%

%/!\/!\/!\/!\/!\/!\/!\/!\/!\/!\/!\/!\/!\/!\/!\/!\/!\/!\/!\/!\/!\/!\/!\/!\%
\section{CONCLUSION}
\label{sec:CONCLUSION}
%/!\/!\/!\/!\/!\/!\/!\/!\/!\/!\/!\/!\/!\/!\/!\/!\/!\/!\/!\/!\/!\/!\/!\/!\%


%% If you have bibdatabase file and want bibtex to generate the
%% bibitems, please use
%%
%%  \bibliographystyle{elsarticle-num} 
%%  \bibliography{<your bibdatabase>}

%% else use the following coding to input the bibitems directly in the
%% TeX file.

%\begin{thebibliography}{00}
\bibliography{biblioArticle1.bib}
\bibliographystyle{elsarticle-num}
%\bibliographystyle{unsrt} 
%\end{thebibliography}


\end{document}
\endinput